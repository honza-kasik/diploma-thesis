\documentclass[
  master,
%  field=inf,
%  printversion,
  biblatex,
%  language=english,
%  font=sans,
  glossaries,
  index
]{kidiplom}

\usepackage{tabularx}

\title{Platformově nezávislý masivně paralelní distribuovaný testovací framework}
\title[english]{Platform agnostic massively parallel distributed testing framework}

\author{Jan Kašík}

\annotation{Cílem této diplomové práce, vypsané společností Red Hat Czech, s.r.o., je analyzovat možnosti v oblasti masivně paralelního distribuovaného testování a vytvořit prototyp frameworku, který adresuje nedostatky a obchází překážky nalezené v existujících řešeních. Práce se skládá ze dvou částí. V první části student provede podrobnou analýzu zaměřenou na funkční schopnosti, použité algoritmy a inženýrské přístupy použíté v existujících systémech. Ve druhé části student vytvoří prototyp frameworku pro multiplatformní synchronizaci kontextu mezi uzly. Framework bude možné  použít k událostmi řízené orchestraci rozmanitých platforem pomocí agentů.
Framework bude distribuovat a orchestrovat agenty pomocí kterých bude exekuovat a synchronizovat různé úlohy. K exekuci a obsluze těchto úloh budou použity zásuvné konektory. Framework bude umožňovat uživateli přidat nové zásuvné konektory pro jiné platformy a bude tak rozšiřitelný.
Framework bude používat konfigurační jazyk pro abstrakci uživatele od úloh samých (doménově specifický nebo obecný jazyk na uvážení studenta). Úroveň abstrakce poskytovaná zásuvnými konektory bude tak vysoká, že uživatel nebude muset používat nativní jazyk dané platformy.
Zdrojový kód práce bude dostupný pod podmínkami licence GNU/GPLv3 nebo kompatibilní.}

\annotation[english]{The goal of this thesis is to analyse open vistas in the field of massively parallel distributed testing and to prototype a framework that addresses shortcomings and overcomes obstacles found in currently existing solutions. The thesis consists of two parts, in the first part, student carries out a thorough analysis focused on operational capabilities, used algorithms and engineering approaches leveraged in currently existing systems. In the second part, student will create a prototype of a framework for multiplatform context synchronization between nodes. It will be possible to use this framework to orchestrate various platforms by its agents in an event driven way.
The framework will distribute and orchestrate agents through which it will be executing and synchronizing various tasks. Pluggable connectors will be used to execute and manipulate these tasks. Framework will allow user to create and add additional pluggable connectors for other platforms to keep it extensible.
Framework will be using configuration language to abstract user from tasks itselves (DSL or GPL – at discretion of student). The level of abstraction, provided by pluggable connectors, should be so high, that user does not have to use platform’s native language.
The final thesis source code will  be available under terms of GNU/GPLv3 licence or compatible.}

\keywords{paralelní a distribuované systémy, testování, open source}
\keywords[english]{parallel and distributed systems, testing, open source}

\thanks{Děkuji, děkuji, děkuji.}

\bibliography{bibliography.bib}

\begin{document}
\maketitle

\noindent\textcolor{red}{\LARGE Upozornění: Následující text je rozpracovaná a (značně) neúplná verze!!!}

\section{Motivation}

Everyday we enjoy perks of distributed systems. From life like examples like network of independent railway stations with trains as a mean of connection between them to practical computer solutions which are build for various reasons. Of which the scalability, high availability and security would be the most frequent.

I will try to bring some of those use cases closer to reader to show, how important distributed systems are and how crucial is to verify that they are working properly.

\subsection{High availablity}

All users of any service want to have an access to it everytime they wish. Failure of a critical service which would cause its inaccessibility when you need it is not an acceptable scenario in most cases. This implies that we need to know how reliable some service is if we want deploy it to some business critical environment. The degree of this reliability is also called availability.

The simplest definition of average avalability $A$ is proportion of expected uptime (mean time between failures) of some service to sum of expected downtime (mean time to restore) and uptime

$$A=\frac{E(uptime)}{E(uptime) + E(downtime)}$$

this gives us a percentage of time during which system performs its required function\cite{bib:reliability}. In this case it is a time during which the service is available to users.

Provider and user usually agreeds on SLA (service level agreement). There are also specified other aspects of service like quality and responsibilities of contracting parties. Table \ref{table:mttr} shows, how agreed availability varies from values common for cheap web hosting service to critical availability typical for realm time systems. Such promised availability is also called {\bf high availability}.

To achieve such availability, multiple approaches are being used. The main one is redundancy. Redundancy in harware is the easiest way how to achieve one. Redundant power supply, redundant storage (e.g. RAID) etc. It just won't help when that one machine blows its single point of failure which is usually mother board. In that case it is clear we need multiple machines working as service providers. Nodes, which would run the same code, providing the exact same service.

There is now multiple nodes providing same service, but how to control to which one users connect without breaking their comfort? We need a top layer mechanism in infrastructure to take the control over "redirection" of users requests (which is very simplified) and detecting failures in lower layer. Although this mechanism becomes new single point of failure and evade this may be challenging, it makes the whole system more reliable. %TODO citation needed.

\begin{table}[]
\centering
\begin{tabular}{lcccc}
Availability & MMTR/year  & MTTR/month & MTTR/week & MTTR/day  \\ \hline
90           & 36.5 d     & 72 h       & 16.8 h    & 2.4 h     \\
99           & 3.65 d     & 7.2 h      & 1.68 h    & 14.4 m    \\
99.5         & 1.83 d     & 3.6 h      & 50.4 m    & 7.2 m     \\
99.9         & 8.76 h     & 43.8 m     & 10.1 m    & 1.44 m    \\
99.99        & 52.56 m    & 4.38 m     & 1.01 m    & 8.64 s    \\
99.999       & 5.26 m     & 25.9 s     & 6.05 s    & 864.3 ms  \\
99.9999      & 31.5 s     & 2.59 s     & 604.8 ms  & 86.4 ms   \\
99.99999     & 3.15 s     & 262.97 ms  & 60.48 ms  & 8.64 ms   \\
99.999999    & 315.569 ms & 26.297 ms  & 6.048 ms  & 0.864 ms  \\
99.9999999   & 31.5569 ms & 2.6297 ms  & 0.6048 ms & 0.0864 ms
\end{tabular}
\caption{Ilustration of system downtimes based on availability agreed in SLA}
\label{table:mttr}
\end{table}

\subsection{Load balancing}

This kind of mechanism is called {\bf load balancer}, since it distributes workload between redundant nodes in cluster. It must be also said that this load balancer needs to detect failures of individual nodes and stop redirecting to those which fail. Verifying funcionality of such load balancer is very complex and is main motivation for my thesis. I will now introduce several examples of load balancers.

%TODO picture here (load balancer proxy, workers, client)

\subsubsection{DNS load balancer}

The simplest way of load balancing is DNS based load balancing. If you are familiar with Domain Name Service, you can skip following two paragraphs which introduce reader into the matter.

The main purpose of DNS is to associate IP address with a hostname. A hostname, which is far more rememberable for human being than, in a worst case, a sequence of hexadecimal symbols divided by colons. When user initites a basic command \kiinlinecode{}{!}{ping} with a hostname as an argument, a program called \kiinlinecode{}{!}{resolver} takes over. For the purpose of an example, all caches will be omitted and the hostname will be \kiinlinecode{}{!}{upol.cz}. The \kiinlinecode{}{!}{resolver} first asks DNS server configured in operating systems, if it knows the IP address of \kiinlinecode{}{!}{upol.cz}. It probably doesn't, but it provides resolver with an address of someone who could -- the root server. \kiinlinecode{}{!}{resolver} asks root server for the address, but it doesn't know it. It only knows who could know it -- the server responsible for \kiinlinecode{}{!}{cz} domain and provides resolver with its address. It sends query to that server once again, asking, who is reponsible for \kiinlinecode{}{!}{upol.cz} domain. After receiving an answer, the \kiinlinecode{}{!}{resolver} sends a final query to that address recieving response like shown in source code \ref{code:dig-upolcz}.

\begin{kicode}{}{code:dig-upolcz}{Answer section of output from dig upol.cz command}
;; ANSWER SECTION:
upol.cz.		86022	IN	A	158.194.230.95
\end{kicode}

Note that the response in source code one provides just one address for \kiinlinecode{}{!}{upol.cz} domain, since this will be subject of further explanation. This is just a short introduction to DNS and it definitely doesn't cover all shortcomings and corner cases.

DNS load balancing is round-robin based. With appropriate DNS server configuration, the answer recieved by \kiinlinecode{}{!}{resolver} looks like the one in source code \ref{code:dig-infupolcz}. There are multiple IP addresses assigned for one hostname. The resolver usually picks the first and the DNS server rotates these addresses in every response so the traffic is equally distributed.

\begin{kicode}{}{code:dig-infupolcz}{Answer section of output from dig inf.upol.cz command}
;; ANSWER SECTION:
inf.upol.cz.		86400	IN	A	158.194.92.6
inf.upol.cz.		86400	IN	A	158.194.80.18
\end{kicode}

Although the principle of this simple system sounds great, it really doesn't work so well. The main issue is caching. In DNS, almost everything is cached with various expiration time anywhere between few days to a week. This causes the fact, that lot of clients which queries DNS server with cached queries within expiration interval, recieve same response. Thus they will try to connect to the same server which can put a significant load on it.

This solution also cannot detect any failures of servers which addresses it provides. There is no protocol for a talk between DNS server and host, so there is no way for DNS server how to find out that the host is not alive anymore \cite{bib:slb}\cite{bib:practicallb}.

\subsubsection{Content delivery network}

Important factor which influences response time of target host is its geographic location in relation to client. CDN or more archaical term Global server load balancing\cite{bib:slb} relies on fact, that the closer the host is to the client, the faster is response from the host from the client's point of view. Not only response can be faster. If the host and client are in the same wider network, the client can usually access to the host on a connection with higher bandwidth, thus downloading content faster\cite{bib:practicallb}.

From historical point of view, in times, when connection was priced by bytes transfered, the internet service providers (ISPs) tried to find a way how to save money by transfering less data through foreign networks. The solution was quite simple and efficient. The ISPs established a mirror (usually a FTP server) of often downloaded large files from those networks and provided it to its clients. This was beneficial for both sides. ISPs saved money and client downloaded desired file faster.

As it was already said, the basis of this system stands on servers in various geographic locations which mirrors their content. As of present, you can pay for service proving funcionality of CDN and use it for static content of your website since building network like this can be and in fact is very expensive.

\subsection{Everyday life distributed systems}

Although verifying the correct funcionality of a load balancer is very important, there are other use cases when confirming precise funcionality of a distributed system is crucial. Such systems are more and more common since they are much easier to scale. In distributed system, when more power is needed, the only thing which has to be done is add an other new node. %TODO more general talk about distributed systems

\subsubsection{Operating-system-level virtualization}

So called containers or less likely heard words written in the heading are isolated instances of operating systems kernels running in user-space. Due to the fact that for the programs running inside them is their environment almost unrecognizable from real environment, are such containers used commonly for virtualization hosting. When comes to security, containers comes handy to, because it is possible to isolate applications in them too. %TODO - more complex use cases, more introduction, concrete technology solutions mentioning?

\subsubsection{Cloud}

Cloud computing based on lending out the server resources is another popular term nowadays.

\subsubsection{Microservices}

Running pieces of code which are loosely connected between each other. A server oriented architecture which promotes writing applications how they should be really written -- modularized, independent and reusable chunks. %TODO - more text

\subsection{mod\_cluster}

%TODO - this probably belongs under HTTP load balancing

The initial comes for this thesis comes from need to test mod\_cluster -- a HTTP based load balancer. As I made introduction to load balanciong before, now I will only make few notes about mod\_cluster and its versaility.

First, let's introduce some neccassary terminology. Client in following text is an end user, who wants to profit from working load balancing. Balancer is a network node which performs load balancing. Worker is a network node which consumes client's requests and generates responses -- often an application server with deployed application. %TODO - often?? [citation needed]

mod\_cluster uses own set of HTTP methods, often called mod\_cluster management protocol (MCMP)\cite{bib:modcluster}, to establish communication between workers and balancer. This channel can be used to transmit load balance factors and application lifecycle events from workers to balancer.

Over other HTTP based load balancers, mod\_cluster has following advantages:

Application life cycle control. Most of other solutions cannot distinguish between request for undeployed application from application server and request for non-existent resource. In both cases, this kind of request ends up with a 404 HTTP response code. In mod\_cluster load balancing environment, workers forwards to balancer application lifecycle events like deploy and undeploy. The balancer then know when it can stop or start routing requests to that worker.\cite{bib:modcluster}

Worker side load balance factor calculation. Traditional HTTP based solution calculate load balance factors on balancer on balancer node. In mod\_cluster environment, these factors are calculated and provided by worker nodes.\cite{bib:modcluster}

Dynamic worker configuration. %TODO

%TODO - complete HAproxy
\begin{table}[]
\centering
\caption{Comparsion of various load balancers}
\label{my-label}
\begin{tabularx}{\linewidth}{X|cccc}
                                        & NGINX                             & mod\_cluster              & mod\_jk          & HAproxy \\ \hline
Dynamic worker configuration            & no\cite{bib:nginxlb}              & yes\cite{bib:modcluster}  & no               &         \\
Load balancer factor provided by worker & no\cite{bib:nginxlb}              & yes\cite{bib:modcluster}  & N/A\footnotemark &         \\
Application life cycle control          & no\footnotemark\cite{bib:nginxlb} & yes\cite{bib:modcluster}  & no               &
\end{tabularx}
\end{table}
\footnotetext{There is no load balancer factor in mod\_jk load balancing}
\footnotetext{It however looks for server's health\cite{bib:nginxlb}}


\section{The output of my work}

%TODO - this is huge

===============================================

\section{Historical solution for distributed testing}

\section{Contemporary solutions for distributed testing}

\subsection{Radar gun}

\section{The problem}

First performance then float to degrading network testing

\section{My contribution/the solution}

\subsection{Openshift/Kubernetes automation}

sunstone, jclouds, xft-cz

\subsection{Traffic control}

creating degraded network in Openshift

\subsection{Automate it/putting it together}

\section{Implementation part}


\printbibliography

\end{document}
